\documentclass[11pt,a4paper,sans]{moderncv}        % possible options include font size ('10pt', '11pt' and '12pt'), paper size ('a4paper', 'letterpaper', 'a5paper', 'legalpaper', 'executivepaper' and 'landscape') and font family ('sans' and 'roman')

% moderncv themes
\moderncvstyle{classic} % style options are 'casual' (default), 'classic', 'oldstyle' and 'banking'
\moderncvcolor{blue} % color options 'blue' (default), 'orange', 'green', 'red', 'purple', 'grey' and 'black'
%\renewcommand{\familydefault}{\sfdefault}         
\nopagenumbers{}

\usepackage[T1]{fontenc}
\usepackage{lmodern}
\usepackage{cmap}

\usepackage[utf8]{inputenc}
\usepackage[magyar]{babel}
\usepackage{fontawesome5}

\usepackage[scale=0.84]{geometry}
\setlength{\hintscolumnwidth}{2cm}                % if you want to change the width of the column with the dates
%\setlength{\makecvtitlenamewidth}{10cm}           % for the 'classic' style, if you want to force the width allocated to your name and avoid line breaks. be careful though, the length is normally calculated to avoid any overlap with your personal info; use this at your own typographical risks...

\newcommand{\weblink}[2]{{\href{#1}{#2}}}

\name{Ádám}{Zsófia}
\title{Mérnökinformatikus $\cdot$ PhD Hallgató $\cdot$}
\extrainfo{
	Budapest, Magyarország \faHome{}\\
	\weblink{mailto:adamzsofi@edu.bme.hu}{adamzsofi@edu.bme.hu} \faEnvelope{}\\
	\weblink{http://github.com/AdamZsofi}{AdamZsofi \faGithub{}}\\
	\weblink{https://www.linkedin.com/in/zs\%C3\%B3fia-\%C3\%A1d\%C3\%A1m-b534341a4/}{ \faLinkedin{}}\\
	\weblink{https://orcid.org/0000-0002-6551-5860}{0000-0002-6551-5860 \faOrcid{}}
}

\setlength{\fboxsep}{0pt}
\photo[70pt][0.3pt]{cropped-square.jpg}
%\quote{Some quote}

\begin{document}

\makecvtitle
\vspace{-3em}

\section{\faGraduationCap{} Tanulmányok}
\cventry{2023--}{Informatika PhD}{Budapesti Műszaki és Gazdaságtudományi Egyetem, Mesterséges Intelligencia és Rendszertervezés Tanszék}{}{}{}
%
\cventry{2022--2023}{Mérnökinformatikus MSc}{Budapesti Műszaki és Gazdaságtudományi Egyetem}{\weblink{https://ftsrg.mit.bme.hu/theta/publications/adamzsMsc2023.pdf}{Szakdolgozat: Extending the Capabilities of the CEGAR Model Checking Algorithm \faFile*[regular]}}{}{}
%
\cventry{2018--2022}{Mérnökinformatikus BSc}{Budapesti Műszaki és Gazdaságtudományi Egyetem}{\weblink{https://ftsrg.mit.bme.hu/theta/publications/adamzsBsc2021.pdf}{Szakdolgozat: Hatékony technikák C programok formális verifikációjához \faFile*[regular]}}{}{}

\section{\faGlobe{} Tapasztalatok}
\cventry{2023 és 2024 Nyár}%
{LMU}{München, Németország}{Research Stay}{}%
{A müncheni Ludwig-Maximilians Egyetem Software Verification and Systems Lab csoportja jól ismert a szoftver verifikáció területén, sok éve folyamatosan vezető kutatási eredményeket adva a területhez. Prof. Dr. Dirk Beyer csoportvezető vendéglátásában két hónapot töltöttem a csoportnál. A BTOR2C eszközzel kapcsolatos témán dolgoztam, hardver ellenőrzés kimeneteként kapott, helyességet bizonyító tanúk validációs módszerein. }
%
\cventry{2022 Nyár}%
{CERN}{Meyrin, CH}{CERN Summer Student}{}%
{A CERN "nyári diák" (Summer Student) programjának keretében két hónapot töltöttem a Beam Department alatti Industrial Control Systems Group csoportnál. PLC verifikációval, elsősorban követelmény formalizációval foglalkoztam.}
%
\cventry{2021 Nyár}%
{thyssenkrupp Components Technology Hungary}{Budapest, HU}{Software Engineering Intern
at thyssenkrupp}{}%
{Statikus analízis szabályok fejlesztése a szoftverfejlesztő csoport házi kódolási szabályai alapján.}
%
\cventry{2019--2023}%
{Budapesti Műszaki és Gazdaságtudományi Egyetem}{Budapest, HU}{Demonstrátor}{}%
{Gyakorlatok tartása, zárthelyi és vizsga feladatok összeállítása és javítása, házi feladat informatikai rendszerének karbantartása több kurzuson is.}
%
\section{\faLightbulb{} Képességek, ismeretek}
\cvitem{Kutatás}{szoftver modellellenőrzés, formális módszerek, CEGAR, eszközfejlesztés, verifikációs portfóliók és algoritmus választási technikák}
\cvitem{Programozás}{Java, C/C++, git, CI, scripting: Python, Bash}
\cvitem{Nyelvtudás}{Magyar (anyanyelv), angol (C1 szint), német (B2 szint)}

\section{\faFile*[regular]{} Válogatott Publikációk}
\cventry{TACAS 2024}%
{Btor2-Cert: A Certifying Hardware-Verification Framework Using Software Analyzers}%
{\weblink{https://link.springer.com/chapter/10.1007/978-3-031-57256-2_7}{\faFile*[regular]}\newline{}
\underline{Zs.~Ádám}, et al}
{}{}{}
%
\cventry{TACAS (SV-COMP) 2024}%
{ConcurrentWitness2Test: Test-Harnessing the Power of Concurrency (Competition Contribution)}%
{\weblink{https://link.springer.com/chapter/10.1007/978-3-031-57256-2_16}{\faFile*[regular]}\newline{}
L.~Bajczi, et al}
{}{}{}
%
\cventry{TACAS (SV-COMP) 2024}%
{EmergenTheta: Verification Beyond Abstraction Refinement (Competition Contribution)}%
{\weblink{https://link.springer.com/chapter/10.1007/978-3-031-57256-2_23}{\faFile*[regular]}\newline{}
L.~Bajczi, et al}
{}{}{}
%
\cventry{NFM 2023}%
{From Natural Language Requirements to the Verification of Programmable Logic Controllers}%
{\weblink{https://link.springer.com/chapter/10.1007/978-3-031-33170-1_21}{\faFile*[regular]}\newline{}
\underline{Zs.~Ádám}, et al}
{}{}{}
%
\cventry{FormaliSE 2022}%
{C for yourself: comparison of front-end techniques for formal verification}%
{\weblink{https://dl.acm.org/doi/abs/10.1145/3524482.3527646?casa_token=U9v1qY6fUH0AAAAA:UjdcqXPYwKOTL5saCQQb4U69GlHSEDtf7RkObTe09CXVOA4Mxo6vHAap6CcUNxMAS9lDkyZnDdU}{\faFile*[regular]}\newline{}
L.~Bajczi, \underline{Zs.~Ádám}, Hajdu, V.~Moln\'ar}
{}{}{}
%
\cventry{TACAS SV-COMP 2022}%
{Theta: portfolio of CEGAR-based analyses with dynamic algorithm selection (Competition Contribution)}%
{\weblink{http://ftsrg.mit.bme.hu/paper-tacas2022-theta/paper.pdf}{\faFile*[regular]}\newline{}
\underline{Zs.~Ádám}, et al}
{}{}{}
%
\cvitem{ORCID}{\weblink{https://orcid.org/0000-0003-2354-1750}{\small 0000-0003-2354-1750}}
\cvitem{MTMT}{\weblink{https://m2.mtmt.hu/gui2/?type=authors&mode=browse&sel=10077295}{Publikációs Lista az MTMT-n, 10077295}}

%\newpage

\section{\faFile*[regular]{} Kontribúciók Nyílt Forráskódú Szoftverekhez}
\cvitem{\weblink{http://github.com/ftsrg/theta/}{Theta}}{CEGAR algoritmus fejlesztései, portfólió stratégiák kutatása, implementálása, C ellenőrző frontend implementálás.}
\cvitem{\weblink{https://gitlab.com/plcverif-oss}{PLCverif}}{A Formal Requirement Elicitation Tool (FRET) eszköz implementációja a PLCverif CERN által fejlesztett PLC ellenőrző eszközbe.}
\cvitem{\weblink{http://github.com/ftsrg/gazer/}{Gazer}}{Fejlesztések, mérések a Gazer BME Kritikus Rendszerek kutatócsoportjánál fejlesztett BMC ellenőrzőhöz}

\section{\faUniversity{} Akadémiai Tevékenységek}
\cvitem{2024}{iFM Subreviewer}
\cvitem{2023}{FormaliSE Subreviewer}
\cvitem{2022}{ISSRE-W Subreviewer}
\cvitem{2021, 2022, 2023, 2024}{SV-COMP (TACAS) Subreviewer}
\cvitem{2024}{SEFM Artifact Evaluation PC}
\cvitem{2022}{CAV Artifact Evaluation Subreviewer}

\section{\faTrophy{} Elnyert Díjak és Ösztöndíjak}
\cvitem{2023}{\weblink{https://ms.sapientia.ro/otdk2023/hu/hirek/eredmenyek}{OTDK első helyezés (Formális módszerek)}}
\cvitem{2022}{\weblink{https://tdk.bme.hu/VIK/beagy1/Absztrakcioalapu-trace-generalas-formalis}{Kari TDK első helyezés (Beágyazott Rendszerek)}}
\cvitem{2021}{\weblink{https://tdk.bme.hu/VIK/ViewPaper/Portfoliobased-runtime-improvements-for}{Kari TDK második helyezés (Szoftver)}}
\cvitem{2022}{"ÚNKP" Ösztöndíj}
\cvitem{2021--2023}{"NFÖD" Ösztöndíj}
\cvitem{2021--2023}{"KBME" Ösztöndíj}

\section{\faBook{} Oktatás}
\cvitem{Tárgyak}{%
	Szoftvertechnikák (németül és magyarul) $\cdot$
	Szoftver~és~Rendszer~Ellenőrzés $\cdot$
	Rendszermodellezés $\cdot$
	Rendszertervezés $\cdot$
	Programozás~Alapjai~1 (németül) $\cdot$
	Digitális~Technikák%
}
\cvitem{Hallgatói Konzultáció}{%
	1 BSc Hallgató (sikeres szakdolgozat és diploma szerzés)
}

\section{\faHandshake{} Önkénteskedés}
%
\cventry{2017--2019}%
{\weblink{https://skool.org.hu/}{Skool}}{Budapest, HU}{Mentor \& Programozás Oktató}{}%
{Az informatika és kapcsolódó területek bemutatása fiatal lányoknak, bátorításuk az ezeken a területeken való továbbtanulásra.}
%
\cventry{2023--2024}%
{\weblink{https://lanyoknapja.hu/}{Lányok Napja}}{Budapest, HU}{Előadó}{}%
{BME VIK kari labor látogatások egyik előadójaként vettem részt ezen az eseményen.}
%
\cventry{2017, 2018, 2024}%
{\weblink{https://www.kutatokejszakaja.hu/}{Kutatók Éjszakája}}{Budapest, HU}{Demo Előadó}{}%
{Az ftsrg által szervezett bemutatón vettem részt, mint előadó.}

\end{document}
