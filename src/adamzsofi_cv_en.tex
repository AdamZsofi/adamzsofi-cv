\documentclass[11pt,a4paper,sans]{moderncv}        % possible options include font size ('10pt', '11pt' and '12pt'), paper size ('a4paper', 'letterpaper', 'a5paper', 'legalpaper', 'executivepaper' and 'landscape') and font family ('sans' and 'roman')

% moderncv themes
\moderncvstyle{classic} % style options are 'casual' (default), 'classic', 'oldstyle' and 'banking'
\moderncvcolor{green} % color options 'blue' (default), 'orange', 'green', 'red', 'purple', 'grey' and 'black'
%\renewcommand{\familydefault}{\sfdefault}         
\nopagenumbers{}

\usepackage[utf8]{inputenc}
\usepackage[magyar]{babel}
\usepackage{fontawesome5}

\usepackage[scale=0.84]{geometry}
\setlength{\hintscolumnwidth}{2cm}                % if you want to change the width of the column with the dates
%\setlength{\makecvtitlenamewidth}{10cm}           % for the 'classic' style, if you want to force the width allocated to your name and avoid line breaks. be careful though, the length is normally calculated to avoid any overlap with your personal info; use this at your own typographical risks...

\newcommand{\weblink}[2]{{\href{#1}{#2}}}

\name{Zsófia}{Ádám}
\title{Computer scientist $\cdot$ PhD Student}
\extrainfo{
	Budapest, Magyarország \faHome{}\\
	\weblink{mailto:adamzsofi@edu.bme.hu}{adamzsofi@edu.bme.hu} \faEnvelope{}\\
	\weblink{adamzsofi.github.io}{Webpage \faGlobe{}}\\
	\weblink{http://github.com/AdamZsofi}{AdamZsofi \faGithub{}}\\
	\weblink{https://www.linkedin.com/in/zs\%C3\%B3fia-\%C3\%A1d\%C3\%A1m-b534341a4/}{Linkedin \faLinkedin{}}\\
	\weblink{https://orcid.org/0000-0002-6551-5860}{0000-0002-6551-5860 \faOrcid{}}
}

\setlength{\fboxsep}{0pt}
\photo[70pt][0.3pt]{cropped-square.jpg}
%\quote{Some quote}

\begin{document}

\makecvtitle
\vspace{-3em}

\section{\faGraduationCap{} Education and Degrees}
\cventry{2023--}{Software Engineering PhD}{Budapest University of Technology and Economics, Department of Artificial Intelligence and Systems Engineering}{}{}{}
%
\cventry{2022--2023}{Software Engineering MSc}{Budapest University of Technology and Economics}{\weblink{https://ftsrg.mit.bme.hu/theta/publications/adamzsMsc2023.pdf}{Thesis: Extending the Capabilities of the CEGAR Model Checking Algorithm \faFile*[regular]}}{}{}
%
\cventry{2018--2022}{Software Engineering BSc}{Budapest University of Technology and Economics}{\weblink{https://ftsrg.mit.bme.hu/theta/publications/adamzsBsc2021.pdf}{Thesis: Efficient Techniques for Formal Verification of C Programs \faFile*[regular]}}{}{}

\section{\faGlobe{} Experience}
\cventry{2023, 2024 Summer}%
{LMU}{Munich, Germany}{Research Stay}{}%
{I spent two months working on validation of correctness proofs of hardware and software modelcheckers at the Software Systems Laboratory under the supervision of Prof. Dr. Dirk Beyer, Nian-Ze Lee and Po-Chun Chien. }
%
\cventry{2022 Summer}%
{CERN}{Meyrin, CH}{CERN Summer Student}{}%
{I learned about and worked on PLC verification, mainly requirement formalization as part of the Industrial Control Systems Group in the Beam Department for two months.}
%
\cventry{2021 Summer}%
{thyssenkrupp Components Technology Hungary}{Budapest, HU}{Software Engineering Intern
at thyssenkrupp}{}%
{Developing rules for C verification with static analyzer based on the in-house coding guidelines.}
%
%\cventry{2019--2023}%
%{Budapest University of Technology and Economics}{Budapest, HU}{Teaching Assistant}{}%
%{Delivering practical lectures, correcting and assembling exams, managing homework IT infrastructure in several courses.}
%
\section{\faLightbulb{} Skills and Interests}
\cvitem{Research}{model checking, formal methods, CEGAR, tool  development, model transformations for verification, portfolios and algorithm selection}
\cvitem{Development}{Java, Kotlin, C/C++, git, CI, Python, Bash}
\cvitem{Languages}{Hungarian (native), English (advanced), German (intermediate)}

%\newpage

\section{\faFile*[regular]{} Selected Publications}
\cventry{TACAS 2024}%
{Btor2-Cert: A Certifying Hardware-Verification Framework Using Software Analyzers}%
{\weblink{https://link.springer.com/chapter/10.1007/978-3-031-57256-2_7}{\faFile*[regular]}\newline{}
\underline{Zs.~Ádám}, et al}
{}{}{}
%
\cventry{TACAS (SV-COMP) 2024}%
{ConcurrentWitness2Test: Test-Harnessing the Power of Concurrency (Competition Contribution)}%
{\weblink{https://link.springer.com/chapter/10.1007/978-3-031-57256-2_16}{\faFile*[regular]}\newline{}
L.~Bajczi, et al}
{}{}{}
%
\cventry{TACAS (SV-COMP) 2024}%
{EmergenTheta: Verification Beyond Abstraction Refinement (Competition Contribution)}%
{\weblink{https://link.springer.com/chapter/10.1007/978-3-031-57256-2_23}{\faFile*[regular]}\newline{}
L.~Bajczi, et al}
{}{}{}
%
\cventry{NFM 2023}%
{From Natural Language Requirements to the Verification of Programmable Logic Controllers}%
{\weblink{https://link.springer.com/chapter/10.1007/978-3-031-33170-1_21}{\faFile*[regular]}\newline{}
\underline{Zs.~Ádám}, et al}
{}{}{}
%
\cventry{FormaliSE 2022}%
{C for yourself: comparison of front-end techniques for formal verification}%
{\weblink{https://dl.acm.org/doi/abs/10.1145/3524482.3527646?casa_token=U9v1qY6fUH0AAAAA:UjdcqXPYwKOTL5saCQQb4U69GlHSEDtf7RkObTe09CXVOA4Mxo6vHAap6CcUNxMAS9lDkyZnDdU}{\faFile*[regular]}\newline{}
L.~Bajczi, \underline{Zs.~Ádám}, Hajdu, V.~Moln\'ar}
{}{}{}
%
\cventry{TACAS SV-COMP 2022}%
{Theta: portfolio of CEGAR-based analyses with dynamic algorithm selection (Competition Contribution)}%
{\weblink{http://ftsrg.mit.bme.hu/paper-tacas2022-theta/paper.pdf}{\faFile*[regular]}\newline{}
\underline{Zs.~Ádám}, et al}
{}{}{}
%

\cvitem{ORCID}{\weblink{https://orcid.org/0000-0003-2354-1750}{\faOrcid{}\small 0000-0003-2354-1750}}
\cvitem{MTMT}{\weblink{https://m2.mtmt.hu/gui2/?type=authors&mode=browse&sel=10077295}{Publication list on MTMT, 10077295}}

\section{\faFile*[regular]{} Open Source Contributions}
\cvitem{\weblink{http://github.com/ftsrg/theta/}{Theta Contributor}}{Algorithmic improvement of CEGAR, researching portfolio strategies, development of a C software model checking frontend.}
\cvitem{\weblink{https://gitlab.com/sosy-lab/software/btor2-cert}{Btor2-Cert Contributor}}{Certified verification of hardware circuits utilizing software model checkers. Translation of software proofs back to the hardware circuit.}
\cvitem{\weblink{https://gitlab.com/plcverif-oss}{PLCverif Contributor}}{Integrated the Formal Requirement Elicitation Tool (FRET) into the PLC verification tool PLCverif developed at CERN.}
\cvitem{\weblink{http://github.com/ftsrg/gazer/}{Gazer Contributor}}{Added features and benchmarked Gazer, a BMC verification tool developed at ftsrg}

\section{\faUniversity{} Academic Services}
\cvitem{2024}{iFM Subreviewer}
\cvitem{2023}{FormaliSE Subreviewer}
\cvitem{2022}{ISSRE-W Subreviewer}
\cvitem{2021, 2022, 2023, 2024}{SV-COMP (TACAS) Subreviewer}
\cvitem{2024}{SEFM Artifact Evaluation PC}
\cvitem{2024}{CAV Artifact Evaluation Subreviewer}

\section{\faTrophy{} Awards and Scholarships}

\cvitem{2023}{"DKÖP" Doctoral Excellence Fellowship Programme Scholarship}
\cvitem{2023}{"ÚNKP" Research Scholarship}
\cvitem{2023}{\weblink{https://ms.sapientia.ro/otdk2023/hu/hirek/eredmenyek}First place at the National Scientific Students' Associations Conference (Formal Methods Category)}
\cvitem{2022}{\weblink{https://tdk.bme.hu/VIK/beagy1/Absztrakcioalapu-trace-generalas-formalis}First place at the Scientific Students' Associations Conference (Embedded Systems Category)}
\cvitem{2021}{\weblink{https://tdk.bme.hu/VIK/ViewPaper/Portfoliobased-runtime-improvements-for}Second place at the Scientific Students' Associations Conference (Software Category)}
\cvitem{2022}{"ÚNKP" Research Scholarship}
\cvitem{2021--2023}{National Academic Scholarship}
\cvitem{2021--2023}{Scholarship of the Faculty of BME-VIK}

\section{\faBook{} Teaching}
\cvitem{Courses}{%
	Software~Techniques (in German and Hungarian) $\cdot$
	Software~and~Systems~Verification~course $\cdot$
	Systems~Modeling $\cdot$
	Software Project Laboratory $\cdot$
	Systems~Engineering $\cdot$
	Basics~of~Programming~1 (in German) $\cdot$
	Digital~Technology%
}
\cvitem{Student Advisor}{%
	Advised 1 BSc Student (successfully graduated)
}
% \cvitem{Students}{Advising 2 BSc Students}

% Next: FormaliSE 24, FASE 25 (both AE PC)

\section{\faHandshake{} Volunteering}
%
\cventry{2017--2019}%
{\weblink{https://skool.org.hu/}{Skool}}{Budapest, HU}{Mentor \& Programming Tutor}{}%
{Teaching young girls on introductory programming workshops to motivate them to take part in IT related fields.}
%
\cventry{2023--2024}%
{\weblink{https://lanyoknapja.hu/}{Girls' STEM Day}}{Budapest, HU}{Presenter}{}%
{I was a presenter at the university faculty's open lab program.}
%
\cventry{2017, 2018, 2024}%
{\weblink{https://www.kutatokejszakaja.hu/}{Researcher's Night}}{Budapest, HU}{Presenter at Demos}{}%
{I presented demos at the demonstration organized by my research group (ftsrg) at the university.}

\end{document}
